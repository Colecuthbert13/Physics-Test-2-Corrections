\documentclass[12pt]{article}
\usepackage{setspace}

\title{Test 2 Corrections\\ 
\large How and why I got questions wrong}
\author{Cole Cuthbert}
\date{November 13, 2015}

\begin{document}

\doublespacing
  \begin{titlepage}
    \vspace*{\fill}
    \begin{center}
      \Huge Test 2 Corrections: \\
      \Large How and why I got questions wrong \\
      \large Cole Cuthbert \\
      \large November 13, 2015
    \end{center}
    \vspace*{\fill}
  \end{titlepage}


\noindent \underline{\textit{Question 2}}
\\ 
\indent 
	The core of this problem lies in isolating each weight and the various forces acting on them. While I found the answer to the acceleration of the system as a whole (${4.67}{m \over s}$), I did did not find the correct tensions. In order to find the tensions, I had based my problem in the equation $T_1=m_1g+m_2a$. I had come up with this equation from thinking that the tension would be the addition of the amount of force with which the two masses at either end ($m_1$ \ and \ $m_2$) were exerting. Following that equation, I plugged in the numbers to find both $T_1$ and $T_2$. By thinking of tension was related to both things attached to it, I tried to account for more things than necessary. The tension can be defined as the net force of \emph{one} object acting on a rope. While I was using both attached masses, the answer required me to look at only one mass and its relation to the rope. A correct way to find tension could be found through knowing that the net force of an object on a rope would be the difference between the tension of the rope and the various forces applied to that object or in our case gravity. This would result in $T_1-m_1g=m_1a$ . Through algebraic manipulation we could then figure out that $T_1=m_1g+m_1a$. This would have been a correct way to find the tension in rope 1, as well as rope 2 (by changing the variables to fit $m_2$).
\\\\
\noindent \underline{\textit{Question 4}}
\\
\indent 
	The issue I had was with the questions regarding the total energy of the system and the escape velocity. I had all of the equations needed to find the answers, however I was confused by the sign of one of the answers so I flipped it, causing me to have another incorrect answer. I also incorrectly input values, leading to an incorrect answer. While I was able to determine the $KE$ of the system ($3.85*10^{28}J$), I did not get the correct $PE$. More specifically, I did get the correct answer on my calculator ($-7.64*10^{28}J$) but I wrote down it down as a positive number. The reason I changed the sign was due to the fact that we were calculating the potential energy of the moon and I thought that because potential energy is stored energy, it could not be negative. Due to this thinking, I ignored that negative and wrote the $PE$ down as a positive. This also effected by subsequent answer, the total $E$ of the system. Since the total energy of a system can be found by adding the potential energy ($PE$) and the kinetic energy ($KE$) I added by $PE$ and $KE$, with a result of ${3.85*10^{28}}+{7.64*10^{28}}={1.15*10^{29}J}$. Because I switched the sign of the potential energy, my answer was very different from the correct answer of ${3.85*10^{28}}+{-7.64*10^{28}}={-3.83*10^{28}J}$. Had I kept the sign as it was, believing that the moon could in fact have negative potential energy, then I could have reached the correct answer. My other issue with question 4 was with finding the escape velocity of something fired from the surface of the moon. I had reached the correct equation needed to find the escape velocity ($V_{escape}={\sqrt{2GM \over r}}$) however I plugged in the wrong values for the mass ($M$) and radius ($r$). When thinking about an object escaping from the moon, I  thought that were an object to leave the surface of the moon and never come back, it would have to escape the moons gravitational pull as well as the earths. Seeing as that the earth is much bigger and therefor exerts more gravitational pull than the moon, I thought to use the mass of the earth as $M$ and the orbital radius of the moon (the distance from the moon to earth) as $r$. Looking back at that question and how I found my answer I see that I had confused both $M$ and $r$. This calculation should have had nothing to do with the earth, as the question was asking about escaping the moon. The values for $M$ and $r$ should be $M=mass \ of \ moon=7.36*10^ {22}kg$ and $r=radius \ of \ moon=1.74*10^{6}m$. With these correct values, the answer is $V_{escape}={\sqrt{2{(6.67*10^{-11})({7.36*10^{22})} \over {1.74*10^{6}}}}}=~2380 \ m/s$ .
\\\\
\noindent \underline{\textit{Question 5}}
\\
\indent
	My issue with this question came from my misuse of the equation \\ %this "\\" is for spacing issue
 $Power=\vec{F}*v$. The third part of question 5 asked us to find the power of the peashooter, so using the equation that I had written down on my LaTeX sheet, I thought that I could multiply the average force of each pea (which I had found one problem earlier) and the final velocity to find the power. This was an incorrect use of that equation due to the nature of the components we were working with. The equation relies upon the object's velocity to be constant throughout the problem (which was not the case for the pea) as well as the force being a vector, which is not the same as the average force of each pea. Having realized that, it is clear that the method I should have used involved the idea that power also equals the work done over time, which can be restated at the change in kinetic energy over the change in time or $P={\Delta{KE}\over{\Delta{t}}}$. It is also known that the change in kinetic energy can be found by $KE_{Final}-KE_{Initial}$, and seeing as the pea started out at rest, the initial $KE=0$. The final $KE$ can be found through the equation ${1\over{2}}mv^2$. This can be placed into the original equation, giving us $P={{{1\over{2}}mv^2}\over{\Delta{t}}}$. Therefore, the correct power of the peashooter was $P={{1\over{2}}{{{(10^{-3})}{(15)}^2}\over{{1/3s}}}}=.338 \ Watts$ .

\end{document}